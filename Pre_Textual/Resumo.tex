% resumo em português
\setlength{\absparsep}{18pt} % ajusta o espaçamento dos parágrafos do resumo
\begin{resumo}
 
A medicina é uma ciência que demonstrou grandes evoluções ao decorrer dos períodos evolutivos, e muitos desafios continuam existindo, tentar evitar o erro humano é um deles. É evidente, que o ato de errar é uma característica humana e estamos sucintos a cometer erros, em um ambiente hospitalar erros humanos podem ocorrer devido a diversos fatores externos (problemas pessoais) e internos (problemas hospitalares). Devido a isso, temos a presença constante de eventos adversos no espaço hospitalar, os mesmos podem ser evitados através de notificações e, consequentemente, tomadas de decisões para solução. Infelizmente, na prática, os casos são poucos notificados devido a diversos fatores que afetam os profissionais da saúde. Não é novidade que diversas ciências estão sendo beneficiadas com a tecnologia, e o processo de solução para os eventos adversos é um ótimo exemplo para empregar a tecnologia, mais uma vez. O principal problema é a ausência de notificações pelos profissionais de saúde devido às formas ultrapassadas de notificar um evento e resistência dos profissionais, com receio de retaliação por parte de autoridades. Para este propósito foi realizada uma Revisão Sistemática dos estudos existentes no contexto e diversos encontros e entrevistas com uma equipe de enfermeiros e acadêmicos em enfermagem para um melhor desenvolvimento da ferramenta. O desafio consiste em tornar a ação prática, através de um processo de notificação virtual e seguro, utilizando recursos para garantir anonimidade durante a notificação de um evento. Consequentemente, com o aumento dos números de notificações, teremos uma melhoria nas tomadas de decisões para solução dos eventos.

 \textbf{Palavras-chave}: Hospitais. Eventos Adversos. Notificações. Aplicativos móveis. API.
 
 
 Segundo a \citeonline[3.1-3.2]{NBR6028:2003}, o resumo deve ressaltar o objetivo, o método, os resultados e as conclusões do documento. A ordem e a extensão destes itens dependem do tipo de resumo (informativo ou indicativo) e do tratamento que cada item recebe no documento original. O resumo deve ser precedido da referência do documento, com exceção do resumo inserido no próprio documento. (\ldots) As palavras-chave devem figurar logo abaixo do resumo, antecedidas da expressão Palavras-chave:, separadas entre si por ponto e finalizadas também por ponto.

 \textbf{Palavras-chave}: latex. abntex. editoração de texto.
\end{resumo}